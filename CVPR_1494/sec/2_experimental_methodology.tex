\section{Experimental Methodology Review}
\label{sec:experimental_methodology}
% Lead Author: Sriya Dhakal
% Team Members: Phaninder Reddy Masapeta, Akhila Ravula, Zezheng Zhang, Scott Weeden

\subsection{Dataset Analysis}
\subsubsection{Training Dataset Diversity and Size}
The experimental design leverages twelve comprehensive medical imaging datasets spanning multiple modalities and anatomical regions. The training data demonstrates exceptional diversity across imaging modalities including CT, MRI (T1-DUAL, T2-SPIR), and PET scans. The datasets encompass diverse anatomical structures ranging from abdominal organs to cardiac structures, providing robust coverage for universal segmentation evaluation.

Key dataset characteristics include AMOS with 500 CT and 100 MRI scans covering 15 abdominal organs including liver, spleen, and kidneys. BCV (AbdomenCT-1K) contributes 100 CT scans focusing on primary abdominal organs. CHAOS provides balanced CT and MRI representation with 20 scans each, specifically targeting liver, kidney, and spleen segmentation tasks.

The dataset composition demonstrates excellent modality diversity across CT, MRI, and PET imaging, ensuring robust cross-modal generalization capabilities. Comprehensive anatomical coverage spanning multiple body regions supports the universal segmentation claims. The preprocessing standardization to 128×128×128 volumes using preprocess.py ensures consistent input dimensions across diverse source datasets.

\subsubsection{Held-out Dataset Selection Rationale}
The evaluation protocol employs seven strategically selected held-out datasets to assess generalization performance. The out-of-distribution evaluation includes ACDC for cardiac segmentation validation, SegTHOR for thoracic organ assessment, and CSI variants for domain shift analysis. The inclusion of MSD Pancreas Tumor and Pelvic1K datasets specifically tests novel class adaptation capabilities on anatomical structures not represented in training data.

This selection strategy effectively tests both domain shift robustness and novel class generalization, two critical aspects of universal medical image segmentation systems.

\subsubsection{Data Preprocessing Appropriateness}
The preprocessing pipeline standardizes all volumes to 128×128×128 resolution using cubic interpolation. While this approach ensures computational efficiency and consistent batch processing, the preprocessing pipeline lacks detailed justification for the chosen resolution parameters and potential impact on fine anatomical structure preservation.

\subsection{Evaluation Protocol}
\subsubsection{Train/Validation/Test Split Methodology}
The experimental design employs a 75% training, 5% validation, and 20% test split across the twelve training datasets. This distribution follows established practices in medical imaging research, although the 5% validation set allocation may be insufficient for robust hyperparameter tuning and early stopping criteria, particularly given the complexity of the multi-dataset training regime.

The episodic training strategy simulates few-shot in-context learning conditions by sampling reference-query pairs during training, effectively preparing the model for inference-time adaptation scenarios.

\subsubsection{Evaluation Metrics Selection}
The primary reliance on Dice Similarity Coefficient represents an appropriate choice for medical segmentation evaluation, providing meaningful overlap assessment between predicted and ground truth segmentations. The Dice metric effectively captures both precision and recall aspects of segmentation quality, making it particularly suitable for medical applications where both false positives and false negatives carry clinical significance.

However, the evaluation framework could benefit from additional complementary metrics such as Hausdorff distance for boundary quality assessment and sensitivity analysis for clinical relevance evaluation.

\subsubsection{Statistical Significance Testing}
The experimental methodology demonstrates a critical deficiency in statistical validation, with no reported significance testing across performance comparisons. This absence undermines confidence in claimed performance improvements and limits the reliability of comparative analysis against baseline methods. Statistical validation through paired t-tests or similar approaches would strengthen the experimental rigor.

\subsection{Baseline Comparisons}
\subsubsection{Fairness of Baseline Implementations}
The comparative evaluation encompasses diverse baseline categories including task-specific models (nnUNet), universal segmentation approaches (CLIP-driven, UniSeg, Multi-Talent), foundation models (SAM variants), and in-context learning methods (SegGPT, UniverSeg, Tyche-IS).

The baseline selection provides comprehensive coverage of relevant methodological approaches, enabling meaningful performance comparison across different paradigms. The experimental design ensures fair comparison by maintaining consistent evaluation protocols and dataset splits across all baseline methods.

\subsubsection{Missing Comparisons}
The baseline comparison could benefit from inclusion of more recent foundation model adaptations and domain-specific in-context learning approaches that have emerged in medical imaging.

\subsubsection{Experimental Controls}
The methodology demonstrates strong experimental control by maintaining consistent data preprocessing, evaluation metrics, and hardware configurations across all compared methods. This standardization ensures that performance differences reflect methodological innovations rather than implementation variations.

The use of identical training data and evaluation protocols across different approaches provides reliable foundations for comparative analysis, though the absence of statistical testing limits the strength of conclusions that can be drawn from observed performance differences.
