\section{Executive Summary}
\label{sec:executive_summary}
% Lead Author: Phaninder Reddy Masapeta (Team Lead)
% Team Members: Sriya Dhakal, Akhila Ravula, Zezheng Zhang, Scott Weeden

\subsection{Overall Assessment}
\textbf{Recommendation: Major Revision}

\textbf{This paper presents Iris, an innovative in-context learning framework for universal medical image segmentation that addresses fundamental limitations in current approaches. The work demonstrates substantial technical merit through its decoupled task encoding architecture and achieves competitive performance across diverse medical imaging datasets. While the core contributions are valuable and the experimental validation is comprehensive, several critical areas require substantial improvement before publication acceptance, particularly in statistical validation, code availability, and analysis of failure modes.}

\subsection{Main Contributions Summary}
The paper presents the following key contributions:
\begin{enumerate}
    \item \textbf{Novel In-Context Learning Architecture}: Introduction of Iris, a framework enabling flexible segmentation using reference image-label pairs without requiring fine-tuning for new tasks or anatomical structures
    \item \textbf{Decoupled Task Encoding Design}: A computational efficiency breakthrough separating task encoding from inference, reducing complexity from O(kmn) to O(k+m) compared to existing methods like UniverSeg
    \item \textbf{Comprehensive Multi-Modal Evaluation}: Extensive validation across twelve diverse medical imaging datasets spanning CT, MRI, and PET modalities, demonstrating robust performance across anatomical regions
    \item \textbf{Multiple Inference Strategy Framework}: Implementation of four distinct inference approaches (one-shot, context ensemble, object-level retrieval, and in-context tuning) providing flexible deployment options for clinical environments
\end{enumerate}

\subsection{Key Strengths}
\begin{itemize}
    \item Architectural Innovation: The decoupled task encoding module represents genuine technical advancement, effectively addressing computational bottlenecks that have limited previous in-context learning approaches in medical imaging.
    \item Comprehensive Experimental Validation: Evaluation encompasses diverse anatomical structures and imaging modalities with consistent performance metrics, achieving approximately 84.52\% average Dice score across in-distribution tasks.
    \item Practical Clinical Relevance: The framework addresses real clinical needs for rapid adaptation to new imaging protocols and anatomical targets without requiring extensive retraining procedures.
    \item Superior Generalization Performance: Demonstrates notable improvement on challenging novel classes, achieving 28.28\% Dice on MSD Pancreas Tumor compared to 11.97\% for best competing methods.
\end{itemize}

\subsection{Major Concerns}
\begin{itemize}
    \item Statistical Validation Deficiency: Complete absence of statistical significance testing undermines the reliability of performance comparisons and limits confidence in claimed improvements.
    \item Code Availability Gap: No repository link or detailed implementation guidelines provided, severely limiting reproducibility and adoption by the research community.
    \item Insufficient Failure Analysis: Limited investigation of failure modes, particularly for challenging cases like extreme domain shifts and small lesion detection, constrains understanding of method limitations.
\end{itemize}

\subsection{Recommendation Justification}
The recommendation for Major Revision reflects the paper's strong foundational contributions balanced against significant methodological gaps. The core technical innovation of decoupled task encoding represents a meaningful advance in medical image segmentation, and the experimental scope demonstrates thorough evaluation across relevant datasets. However, the absence of statistical validation, missing code availability, and limited failure analysis constitute critical deficiencies that must be addressed for publication at a top-tier venue. With these revisions, the work would make a valuable contribution to the field by providing both theoretical advances and practical solutions for clinical medical imaging applications.
