\section{Presentation and Clarity}
\label{sec:presentation_clarity}
% Lead Author: Scott Weeden
% Team Members: Phaninder Reddy Masapeta, Sriya Dhakal, Akhila Ravula, Zezheng Zhang

\subsection{Paper Organization}
\subsubsection{Logical Flow and Structure}
The manuscript follows a well-organized structure that guides readers through the research systematically. The introduction effectively establishes the problem context of medical image segmentation challenges, particularly the limitations of task-specific models when encountering diverse anatomical structures and imaging modalities. The related work section provides appropriate background on existing approaches, including universal models, foundation models, and in-context learning methods. The methodology section clearly explains the Iris architecture, including the decoupled task encoding approach and multiple inference strategies. The experimental section comprehensively evaluates performance across diverse datasets, and the results analysis provides thorough validation of the proposed approach.

The logical progression from problem identification through solution development to comprehensive evaluation demonstrates clear scientific reasoning and effective manuscript organization.

\subsubsection{Section Balance}
The manuscript demonstrates generally appropriate section allocation, with the methodology receiving adequate space for technical details and the experimental section providing comprehensive evaluation coverage. The introduction effectively motivates the research problem without excessive background material. The related work section efficiently positions the work within existing literature. However, the discussion section appears relatively brief given the complexity of the proposed approach and the breadth of experimental findings. The discussion section could benefit from expansion to more thoroughly analyze implications, limitations, and future research directions.

\subsubsection{Abstract and Introduction Effectiveness}
The abstract effectively summarizes the key contributions, including the novel in-context learning framework, decoupled task encoding architecture, and comprehensive experimental validation across diverse medical imaging datasets. The introduction successfully establishes the clinical importance of medical image segmentation and clearly articulates the limitations of current approaches. However, the introduction could strengthen motivation for specific architectural choices, particularly the rationale for decoupled task encoding and the selection of transformer-based components.

\subsection{Writing Quality}
\subsubsection{Clarity of Technical Descriptions}
The technical exposition demonstrates solid clarity in explaining complex architectural components. The task encoding module explanation effectively conveys the dual-stream approach combining foreground and contextual feature processing. The mathematical formulation properly extends traditional segmentation notation to in-context learning scenarios. The description of multiple inference strategies clearly differentiates between one-shot inference, context ensemble, object-level retrieval, and in-context tuning approaches.

Certain mathematical notation elements could benefit from more detailed explanation, particularly the bidirectional cross-attention mechanism and the specific implementation of pixel shuffle operations.

\subsubsection{Grammar and Language}
The manuscript maintains professional academic writing standards throughout most sections. The technical terminology remains consistent across sections, and the sentence structure effectively communicates complex concepts. Minor grammatical inconsistencies appear in several locations, including occasional unclear pronoun references and inconsistent hyphenation patterns. The overall writing quality supports clear communication of technical contributions without significant language barriers.

\subsubsection{Terminology Consistency}
The manuscript generally maintains consistent terminology, though several terms warrant clarification for improved precision. The distinction between context and reference requires more explicit definition, particularly given their interchangeable usage in different sections. Similarly, task encoding and task embedding appear to refer to related but potentially distinct concepts that would benefit from clearer differentiation. A terminology table in the appendix would enhance clarity and support reproducibility efforts.

\subsection{Figures and Tables}
\subsubsection{Quality and Readability}
The visual presentation demonstrates strong quality across most elements. Figure 1 provides excellent architectural overview, clearly illustrating the relationship between task encoding and mask decoding components. The performance comparison charts effectively communicate quantitative results across different inference strategies. The t-SNE visualization convincingly demonstrates learned anatomical relationships within the task embedding space.

Several tables suffer from font size limitations that may impair readability in print formats. The information density in certain tables creates challenges for rapid comprehension of key results.

\subsubsection{Caption Completeness}
Most figure and table captions provide appropriate context for understanding the presented information. The captions generally offer sufficient detail to support independent interpretation without requiring extensive reference to the main text. However, certain table captions lack essential details about experimental setup and evaluation protocols that would enhance reproducibility.

\subsubsection{Information Density}
The tables present substantial information efficiently, though the density occasionally challenges readability. Consideration should be given to splitting complex tables into multiple components or relocating detailed information to supplementary materials while maintaining key results in the main manuscript.

\subsection{Supplementary Material}
\subsubsection{Completeness and Usefulness}
The supplementary material provides valuable additional detail supporting the main manuscript claims. Implementation details receive comprehensive treatment, supporting reproducibility efforts. The additional experimental results enhance understanding of method performance across diverse scenarios.

\subsubsection{Integration with Main Paper}
The relationship between main manuscript content and supplementary material generally supports effective information organization. However, certain implementation details currently relegated to supplementary material appear sufficiently important for inclusion in the main manuscript, particularly those essential for reproducibility. The balance between main paper focus and comprehensive documentation could benefit from reconsideration to ensure that essential technical details remain accessible to readers.
