\section*{Minor Issues}
\label{sec:minor_issues}
% Lead Author: Scott Weeden
% Team Members: Phaninder Reddy Masapeta, Sriya Dhakal, Akhila Ravula, Zezheng Zhang

\subsection*{Typos and Grammatical Errors}
Several minor grammatical and typographical issues require attention to enhance the manuscript's professional presentation. These corrections would improve readability and eliminate potential distractions for reviewers and readers.

The abstract requires punctuation adjustments where several sentences lack appropriate comma placement after introductory clauses. The mathematical notation demonstrates inconsistencies between bold and standard formatting for feature vectors, particularly in the task encoding equations where feature representations alternate between standard and bold formatting without clear rationale.

Several instances of repeated terminology appear within individual paragraphs, suggesting the need for careful proofreading to eliminate redundancy. Caption formatting requires standardization, with some figure captions missing terminal periods while others include them inconsistently throughout the document.

The distinction between possessive and contractive forms appears incorrectly in several locations throughout the methodology section. Table formatting shows inconsistent italicization of technical terms that should follow uniform style guidelines established for the publication venue.

The manuscript contains several run-on sentences, particularly in the results discussion section, that would benefit from division into shorter, more readable statements. These grammatical refinements would enhance clarity without affecting the technical content or experimental conclusions.

\subsection*{Formatting Issues}
Multiple formatting inconsistencies affect the manuscript's visual presentation and readability standards expected for publication. Table alignment problems appear throughout the results section, with column headers not properly centered over their corresponding data values, creating visual confusion when interpreting quantitative results.

Figure quality issues include arrows and connecting lines that may be difficult to perceive in printed versions, particularly in the architectural diagrams where visual clarity is essential for understanding the proposed methodology. Mathematical equation spacing requires adjustment in several locations where crowded notation impairs readability and comprehension.

Section numbering demonstrates inconsistencies in the methodology chapter, where subsection hierarchies do not follow standard academic formatting conventions established for computer vision publications. The reference formatting lacks uniformity, with missing page numbers for several journal articles and inconsistent abbreviation standards for conference names.

Algorithm presentation would benefit from line numbering to facilitate discussion and reproduction by other researchers. Page layout issues include orphaned headings where section titles appear at the bottom of pages without accompanying content, creating poor visual flow.

The supplementary material contains figure reference mismatches that create confusion when readers attempt to locate specific visualizations referenced in the main text. These cross-reference errors impede navigation between the main manuscript and supporting materials.

\subsection*{Reference Errors}
The bibliography requires comprehensive revision to ensure accuracy and consistency across all citations according to established academic standards. Several references contain venue errors where conference proceedings are incorrectly attributed to different venues than where they actually appeared, creating confusion about the publication history of cited work.

Missing author names in multiple citations create incomplete attribution that fails to meet standard academic citation requirements and may cause difficulties for readers attempting to locate the original sources. ArXiv reference formatting shows inconsistencies in year attribution and identification number inclusion, suggesting the need for systematic review.

Conference name abbreviations vary throughout the reference list, with some venues referred to by full names while others use abbreviated forms without clear rationale for the distinction. Citation style inconsistencies appear between different reference types, suggesting the need for systematic formatting review using standard academic style guidelines.

Several preprint papers that have since been published in peer-reviewed venues retain their preprint citations rather than updated publication information, potentially misleading readers about the peer review status of cited work. URL formatting in footnotes requires verification, as several links appear to be broken or incorrectly formatted.

The reference ordering and alphabetization follow inconsistent rules that should be standardized according to the target venue's specific requirements for citation formatting and bibliography organization.

\subsection*{Other Minor Issues}
Additional minor issues affect the manuscript's consistency and accessibility standards expected for publication in a premier venue. The terminology switches between abbreviated and full forms without clear guidelines for when each form should be used, creating inconsistency in technical presentation.

Spelling conventions vary between different regional standards, requiring standardization throughout the document to maintain professional consistency. Mathematical symbol definitions require consistent introduction when symbols first appear in the text, following established conventions for mathematical notation in computer vision literature.

Acronym usage follows inconsistent patterns where some abbreviations are defined upon first use while others appear without proper introduction, potentially confusing readers unfamiliar with specific technical terminology. Figure color schemes present potential accessibility concerns for readers with color vision differences.

The visual design should accommodate accessibility standards to ensure universal readability across different viewing conditions. Capitalization patterns in headings and captions demonstrate inconsistencies that should follow standard title case conventions established for academic publications.

Font sizing in tables and figures requires review to ensure readability across different viewing conditions and print formats, as some technical content may become illegible in certain publication formats. Some technical terms appear in varying font styles without clear rationale for the formatting differences, suggesting the need for systematic style review.

These minor corrections, while individually small, collectively contribute to the professional presentation expected for publication in a top-tier venue. Addressing these issues would demonstrate attention to detail and enhance the overall manuscript quality without requiring changes to technical content or experimental methodology, supporting the goal of achieving publication readiness through comprehensive revision.
