\section{Minor Issues}
\label{sec:minor_issues}
% Lead Author: Scott Weeden

\subsection{Typos and Grammatical Errors}
Several minor grammatical and typographical issues require attention to enhance the manuscript's professional presentation. These corrections would improve readability and eliminate potential distractions for reviewers and readers.

Specific corrections needed include adjusting punctuation in the abstract where several sentences lack appropriate comma placement. The mathematical notation demonstrates inconsistencies between bold and standard formatting for feature vectors, particularly in the task encoding equations where $F_s$ and $\mathbf{F}_s$ are used interchangeably. Several instances of repeated terminology appear within individual paragraphs, suggesting the need for careful proofreading to eliminate redundancy.

Caption formatting requires standardization, with some figure captions missing terminal periods while others include them inconsistently. The distinction between "its" and "it's" appears incorrectly in several locations throughout the methodology section. Table formatting shows inconsistent italicization of technical terms that should follow uniform style guidelines.

The manuscript contains several run-on sentences, particularly in the results discussion section, that would benefit from division into shorter, more readable statements. These grammatical refinements would enhance clarity without affecting the technical content.

\subsection{Formatting Issues}
Multiple formatting inconsistencies affect the manuscript's visual presentation and readability. Table alignment problems appear throughout the results section, with column headers not properly centered over their corresponding data values. Figure quality issues include arrows and connecting lines that may be difficult to perceive in printed versions, particularly in the architectural diagrams.

Mathematical equation spacing requires adjustment in several locations where crowded notation impairs readability. Section numbering demonstrates inconsistencies in the methodology chapter, where subsection hierarchies do not follow standard academic formatting conventions.

The reference formatting lacks uniformity, with missing page numbers for several journal articles and inconsistent abbreviation standards for conference names. Algorithm presentation would benefit from line numbering to facilitate discussion and reproduction.

Page layout issues include orphaned headings where section titles appear at the bottom of pages without accompanying content. The supplementary material contains figure reference mismatches that create confusion when readers attempt to locate specific visualizations.

\subsection{Reference Errors}
The bibliography requires comprehensive revision to ensure accuracy and consistency across all citations. Several references contain venue errors where conference proceedings are incorrectly attributed to different venues than where they actually appeared. Missing author names in multiple citations create incomplete attribution that fails to meet standard academic citation requirements.

ArXiv reference formatting shows inconsistencies in year attribution and identification number inclusion. Conference name abbreviations vary throughout the reference list, with some venues referred to by full names while others use abbreviated forms without clear rationale for the distinction.

Citation style inconsistencies appear between different reference types, suggesting the need for systematic formatting review using standard academic style guidelines. Several preprint papers that have since been published in peer-reviewed venues retain their preprint citations rather than updated publication information.

URL formatting in footnotes requires verification, as several links appear to be broken or incorrectly formatted. The reference ordering and alphabetization follow inconsistent rules that should be standardized according to the target venue's requirements.

\subsection{Other Minor Issues}
Additional minor issues affect the manuscript's consistency and accessibility. The terminology switches between "3D" and "three-dimensional" without clear guidelines for when each form should be used. Spelling conventions vary between British and American English, requiring standardization throughout the document.

Mathematical symbol definitions require consistent introduction when symbols first appear in the text. Acronym usage follows inconsistent patterns where some abbreviations are defined upon first use while others appear without proper introduction.

Figure color schemes present potential accessibility concerns for readers with color vision differences. The visual design should accommodate colorblind-friendly palettes to ensure universal accessibility. Capitalization patterns in headings and captions demonstrate inconsistencies that should follow standard title case conventions.

Font sizing in tables and figures requires review to ensure readability across different viewing conditions and print formats. Some technical terms appear in varying font styles without clear rationale for the formatting differences.

These minor corrections, while individually small, collectively contribute to the professional presentation expected for publication in a top-tier venue. Addressing these issues would demonstrate attention to detail and enhance the overall manuscript quality without requiring changes to technical content or experimental methodology.
