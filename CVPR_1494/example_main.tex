\documentclass[conference]{IEEEtran}

\title{Peer Review of "Show and Segment: Universal Medical Image Segmentation via In-Context Learning"}

\author{
    Phaninder Masapeta \\
    College of Engineering - Texas Tech\\
    Email: phaninder.masapeta@ttu.edu
}

\begin{document}

\maketitle

\begin{abstract}
This review evaluates the strengths and weaknesses of the paper "Show and Segment: Universal Medical Image Segmentation via In-Context Learning." The manuscript presents a novel approach for universal segmentation using in-context learning, demonstrating strong results on multiple datasets.
\end{abstract}

\section{Summary}
The authors propose a universal segmentation framework leveraging in-context learning. The method is evaluated on several medical imaging benchmarks, showing state-of-the-art performance.

\section{Strengths}
\begin{itemize}
    \item Novel use of in-context learning for segmentation.
    \item Comprehensive experiments on diverse datasets.
    \item Clear writing and well-organized presentation.
\end{itemize}

\section{Weaknesses}
\begin{itemize}
    \item Limited discussion on computational cost.
    \item Ablation studies could be more thorough.
\end{itemize}

\section{Suggestions}
\begin{itemize}
    \item Include more details on model efficiency.
    \item Add experiments on additional modalities.
\end{itemize}

\section{Conclusion}
The paper makes a valuable contribution to medical image segmentation. Addressing the above points would further strengthen the work.

\bibliographystyle{IEEEtran}
\bibliography{main}
\end{document}